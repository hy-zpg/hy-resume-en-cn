\documentclass[11pt,a4paper]{moderncv}
%!TEX program = xelatex
\usepackage[UTF8]{ctex}
\usepackage{xeCJK}
\usepackage{fontspec,xunicode,xltxtra}
\defaultfontfeatures{Scale=MatchLowercase}
\setmainfont[Numbers=OldStyle,Mapping=tex-text]{Times New Roman}
\setsansfont[Mapping=tex-text]{Arial}
\setmonofont{Courier New}
\usepackage[BoldFont,SlantFont,CJKchecksingle,CJKnumber,CJKtextspaces]{xeCJK}
% \setCJKmainfont[BoldFont={Adobe Heiti Std},ItalicFont={Adobe Kaiti Std}]{Adobe Song Std}
% \setCJKsansfont{Adobe Heiti Std}
\setCJKmonofont{Adobe Fangsong Std}
\punctstyle{hangmobanjiao}
\defaultfontfeatures{Mapping=tex-text}
\XeTeXlinebreaklocale "zh"
\XeTeXlinebreakskip = 0pt plus 1pt minus 0.1pt
\usepackage{xcolor}
\linespread{1.2}
\moderncvtheme[blue]{classic}
\usepackage[scale=0.95]{geometry}
\AtBeginDocument{\recomputelengths}
\setCJKfamilyfont{name}{Adobe Kaiti Std}
\newcommand\name{\CJKfamily{name}}
\familyname{洪燕}
% \title{求职意向:算法工程师}
\mobile{手机: 13262908839}
\email{邮箱: yanhong@gmail.com}
\homepage{https://www.aiforall.pro}
%\address{详细住址:成都市高新区(西区)西源大道2006号}{邮编:611731}
\photo[90pt]{pic.jpg}
% \quote{\textit{求职意向:算法工程师}}
%----------------------------------------------------------------------------------
%            content
%----------------------------------------------------------------------------------
\begin{document}
\maketitle
\vspace{-3em}      %缩小段落的间距

\section{教育背景}
\cventry{2019/04 -- 2022/04}{博士学位}{上海交通大学}{上海}{成绩:10\%}{计算机科学与技术(研究方向:计算机视觉)}
\cventry{2016/09 -- 2019/03}{硕士学位}{上海交通大学}{上海}{成绩:30\%}{信息与通信工程(研究方向:室内定位,计算机视觉)}
\cventry{2012/09 -- 2016/06}{学士学位}{深圳大学}{深圳}{成绩:1\%}{电子信息工程}
\vspace{-1em}

\section{专业技能}
\cvline{主要技能}{$\bullet$ 学术能力:撰写论文3篇;专利1项。}
\cvline{}{$\bullet$ 编程语言:掌握Python; 用过C/C++, Matlab等。}
\cvline{}{$\bullet$ 深度学习库:掌握Tensorflow,Pytorch; 熟悉Keras, Caffe等。}
\cvline{}{$\bullet$ 算法:掌握神经网络(CNN,RNN 等)数学原理与基础应用,数据结构算法以及图像处理基本算法等。}
%\cvline{}{$\bullet$ 机器学习:用过逻辑回归、随机森林、GBRT、SVD++等。}
% \cvline{}{$\bullet$ 分布式:理解HDFS;写过MapReduce程序;用过Hive。}
%\cvline{}{$\bullet$ 图像处理:熟悉图像识别的基本方法。}
\smallskip
\cvline{英语水平}{$\bullet$  英语六级:500。}
\vspace{-1em}      %缩小段落的间距

\section{研究经历}
\cvline{2019/03 -- 现在}{\textbf{少示例图片生成}}
\cvline{}{$\bullet$ 简要描述:针对少示例样本,利用meta-learning从少示例样本中学习关键特征信息,再利用对抗生成网络模拟创作过程,生成示例相似图片。}
\cvline{}{$\bullet$ 职责:项目负责人。总体方案设计;meta-learning 算法设计;生成对抗网络设计。}
\cvline{2018/03 -- 2019/02}{\textbf{异构数据的多任务学习}}
\cvline{}{$\bullet$ 简要描述:针对异构标签数据集,利用伪标签辅助多任务网络挖掘任务间的相关性。并且从伪标签置信度、分布密度以及异构数据集之间的分布差异对伪标签进行筛选。项目代码详见 \href{https://github.com/bcmi/multi-task-learning}{Multi-task-learning}}
\cvline{}{$\bullet$ 职责:项目负责人。总体方案设计;伪标签筛选算法;多任务网络设计;撰写\textbf{1篇论文}。}
\cvline{2017/09 -- 2018/02}{\textbf{2D视频信息的深层次情感动作识别}}
\cvline{}{$\bullet$ 简要描述:现有动作识别多考虑全局的动作信息,没有将表情信息融入到动作中来传达深层次情感。我们用CNN捕获全局动作与局部表情空间特征(双流CNN);利用 bi-directional LSTM 处理提取的帧图空间特征。项目代码详见 \href{https://github.com/hy-zpg/bimodal-sentiment-analysis}{Bimodal-sentiment-analysis}}
\cvline{}{$\bullet$ 职责:项目负责人。系统整体设计,多模态特征融合算法。 }
\cvline{2016/09 -- 2017/08}{\textbf{深度置信网络的室内WIFI定位}}
\cvline{}{$\bullet$  简要描述:室内 WIFI 信号传播易受干扰,传统算法不易建模。研究深度置信网络的数学原理(DBN);研究生成性网络 DBN 概率特性;采集 WIFI 数据进行训练 DBN 网络,用径向基函数(RBF)计算 WIFI 数据重构的概率来进行定位。项目代码详见 \href{https://github.com/hy-zpg/WIFI-fingerprint-indoor-localization}{WIFI-fingerprint-indoor-localization}}
\cvline{}{$\bullet$ 职责:项目负责人。模型设计;定位算法设计;建立对应测试数据集。撰写\textbf{2篇论文},\textbf{1篇专利}。 }
\vspace{-1em}      %缩小段落的间距

\section{其他经历}
\cvline{多视角目标检测}{$\bullet$ 利用Yolov3目标检测框架提取候选物体特征,通过metric distance来测量候选物体与目标物体的特征距离,通过编码的特征距离匹配目标物体。}
\cvline{微表情识别}{$\bullet$ 上海地铁吉祥物畅畅机器人,对来访的领导进行识别并完成微笑、调皮等系列表情动作。我们需要利用摄像头 捕获的视频信息进行准确的目标VIP识别,并指导机器人完成表情变化。}
\vspace{-1em}      %缩小段落的间距

\section{科研成果}
\cventry{2019}{Beyond Preserving Information: Multi-Task Learning with Disjoint Datasets}{Computer vision}{}{}{}
\cventry{2018}{Indoor Fingerprint Location Technology Based on Deep Belief Network}{Indoor localization}{}{}{}
\cventry{2018}{WiDeep: Learning Featured Fingerprints with DBN framework for Indoor Localization}{Indoor localization}{}{}{}
\cventry{2017}{A WiFi Indoor Positioning System Based on Deep Learning}{Indoor localization}{}{}{}
\vspace{-1em} 

\section{所获奖励}
\cventry{2016 -- 2019}{团支书、二等奖学金}{上海交通大学}{}{}{}
\cventry{2016}{优秀毕业生}{深圳大学}{}{}{}
\cventry{2013 -- 2016}{学业特等奖、优秀班级干部、国家励志奖学金}{深圳大学}{}{}{}
\cventry{2012}{新生奖学金}{深圳大学}{}{}{}
\vspace{-1em}      %缩小段落的间距



% \section{自我评价}
% \cvline{}{$\bullet$通过项目和竞赛学习到了一种方法:理论->实践->理论(论文)。}
% \cvline{}{$\bullet$ 性格开朗,热爱技术,有较强的自学能力和团队协作能力。}
\end{document}