\documentclass[10pt,colorlinks,linkcolor=true,urlcolor=gray]{moderncv}

\moderncvtheme[blue, modroman]{classic}
\usepackage{cv_kleinschmidt}
\usepackage{color}


\addbibresource{kleinschmidt.bib}

\name{Hong}{Yan}
\email{yanhong.sjtu@gmail.com}
\homepage{https://www.aiforall.pro}
\social[github]{https://github.com/hy-zpg}
% \social[twitter]{@kleinschmidt}

\begin{document}

\makecvtitle

\section{Education}
\label{sec:education}

\cventry{2019--Now}{Ph.D. Computer Science and Technology}{Shanghai Jiao Tong University}{Shanghai, China}{}{Majoring in computer vision, advised by Li Niu  \textcolor[rgb]{0.5,0,0}{(\emph{http://bcmi.sjtu.edu.cn/home/niuli/})} in BCMI lab  \textcolor[rgb]{0.5,0,0}{(\emph{http://bcmi.sjtu.edu.cn})}}

\cventry{2016--2019}{M.S. Information Engineering}{Shanghai Jiao Tong University}{Shanghai, China}{}{Majoring in indoor localization and computer vision, advised by Peilin Liu  \textcolor[rgb]{0.5,0,0}{(\emph{http://bat.sjtu.edu.cn/zh/people-zh/})} in BATC lab  \textcolor[rgb]{0.5,0,0}{(\emph{http://bat.sjtu.edu.cn})}}

\cventry{2012--2016}{B.S. Information Engineering}{Shenzhen University}{Shenzhen, China}{}{Majoring in electronic information engineering, {GPA:3.90/4.0, 4/385}}

% (\emph{https://github.com/bcmi/multi-task-learning})
% (\emph{https://github.com/hy-zpg/bimodal_sentiment_analysis})
% (\emph{https://github.com/hy-zpg/WIFI_fingerprint_indoor_localization})
% This project mainly leverages few-shot images to generate new concepts

\section{Research Experiments}
\label{sec:research}
\cventry{2019--Now}{Few-shot image generation}{Computer vision}{Meta learning, generative adversarial network}{}{This research project mainly leverages few-shot images to generate new concepts.}

\cventry{2018--2019}{Multi-task learning with disjoint datasets}{Computer vision}{Multi-task learning, semi-supervised learning}{}{This research project \textcolor[rgb]{0.5,0,0}{(\emph{https://github.com/bcmi/multi-task-learning})} aims to improve multi-task performance with disjoint datasets as input.}

\cventry{2017--2018}{Deep sentiment analysis by combining facial expression and action}{Computer vision}{Multi-task learning}{}{This research project  \textcolor[rgb]{0.5,0,0}{(\emph{https://github.com/hy-zpg/bimodal-sentiment-analysis})} targets at combining facial expression and action to analyze deep sentiment.}


\cventry{2016--2017}{Indoor localization with WiFi fingerprint}{Signal processing}{indoor localization}{}{This research project  \textcolor[rgb]{0.5,0,0}{(\emph{https://github.com/hy-zpg/WIFI-fingerprint-indoor-localization})} mainly focuses on improving WIFI-based indoor localization accuracy with deep belief networks.}



\section{Project Experience}
\label{sec:project}
\cventry{2018-2018}{Micro-emotion recognition}{Computer vision}{Classification}{}{}
\cventry{2017-2018}{Smart home}{Computer vision}{Classification}{}{}
\cventry{2017-2018}{Multi-view object recognition}{Computer vision}{Regression, classification}{}{}
\cventry{2016-2017}{Indoor Localization}{Signal process}{Indoor localization}{}{}





% \section{Appointments}
% \label{sec:experience}

% \cventry{2018}{Assistant Professor of Psychology}{Rutgers University, New Brunswick}{}{}{}
% \cventry{2016--2018}{C.V. Starr Fellow}{Princeton Neuroscience Institute}{}{}{}
% \cventry{2009--2010}{Baggett Fellow}{University of Maryland, Linguistics Department}{}{}{Post-baccalaureate research fellowship. Advised by Bill Idsardi.}
% \cventry{2006--2009}{Research Assistant}{Williams College, Department of Psychology}{}{}{Supervised by Safa Zaki.}

\section{Publications}
\label{sec:publications}
\cventry{2019}{Beyond Preserving Information: Multi-Task Learning with Disjoint Datasets}{Computer vision}{}{}{}
\cventry{2018}{Indoor Fingerprint Location Technology Based on Deep Belief Network}{Indoor localization}{}{}{}
\cventry{2018}{WiDeep: Learning Featured Fingerprints with DBN framework for Indoor Localization}{Indoor localization}{}{}{}
\cventry{2017}{A WiFi Indoor Positioning System Based on Deep Learning}{Indoor localization}{}{}{}
% \nocite{*}
% \printbibliography[heading=subbibliography,title={Papers},type=article, check=inpress]
% \printbibliography[heading=subbibliography,heading=none,type=article, check=nopubstate]

% \printbibliography[heading=subbibliography,title={Submitted and in preparation},
%                    nottype=inproceedings, check=inprepsubmitted]

% \printbibliography[heading=subbibliography,title={Conference Proceedings Papers},type=inproceedings]

% \printbibliography[heading=subbibliography,title={Presentations},
%                    type=unpublished, check=notinprepsubmitted]



% \section{Invited Talks}

% \cventry{2019}{Cognition and Language Workshop}{Stanford University}{May 9, 2019}{}{}
% \cventry{}{Cognitive Science Center}{Williams College}{April 11, 2019}{}{}
% \cventry{2018}{Cognitive Brown Bag series}{Villanova University}{February 8, 2018}{}{}
% \cventry{2017}{Psychology Dept. Perception/Action/Cognition seminar}{University of Connecticut}{October 30, 2017}{}{}
% \cventry{2014}{New Zealand Institute of Language, Brain, and Behavior Seminar}{University of Canterbury}{December 9, 2014}{}{}
% \cventry{}{Cognition and Language Workshop}{Stanford University}{October 2, 2014}{}{}
% \cventry{}{Psychology Department Seminar}{Carnegie Melon University}{July 22, 2014}{}{}
% \cventry{}{Language Processing Brown Bag Seminar}{Beckman Institute, University of Illinois}{March 6, 2014}{}{}
% \cventry{}{Cognitive Science Colloquium}{Tufts University}{February 13, 2014}{}{}

\section{Teaching}
\label{sec:teaching}

\cventry{Spring 2017}{Digital Circuits}{Shanghai Jiao Tong University}{}{}{}
% \cventry{Fall 2018}{Cognitive Science 412: Advanced Topics II}{Rutgers New Brunswick}{}{}{}
% \cventry{February 2014}{Guest Lecturer: Introduction to Cognitive Science}{Tufts University}{}{}{Presented lecture on computational modeling in the cognitive sciences, with an emphasis on statistical (Bayesian) models. }
% \cventry{Spring 2013}{Teaching assistant: Foundations of Cognitive Science}{University of Rochester}{}{}{Lectured, conducted recitations, designed assessments, and graded for large introductory class for the Brain and Cognitive Science Department.}
% \cventry{Spring 2012}{Teaching assistant: Cognition}{University of Rochester}{}{}{Lectured, designed assessments, and graded for introductory cognitive neuroscience course.}
% \cventry{Spring 2011}{Teaching assistant: Cognition}{University of Rochester}{}{}{Lectured, designed assessments, and graded for introductory cognitive neuroscience course.}

% \cventry{Spring 2009}{Teaching assistant: Ergodic Theory}{Williams College}{}{}{Graded problem sets and conducted group work sessions for senior seminar.}

\section{Honors and Awards}
\label{sec:honors-awards}


\cvlistitem{\textbf{League secretary, Second-class scholarship}, Shanghai Jiao Tong University, 2016-2018}
\cvlistitem{\textbf{Outstanding Graduate Award}, Shenzhen University, 2016}
\cvlistitem{\textbf{Outstanding Bachelor Dissertations/theses }, Shenzhen University, 2016}

\cvlistitem{\textbf{National Endeavor Fellowship}, Shenzhen University, 2012-2016}
\cvlistitem{\textbf{Primary Scholarship}, Shenzhen University, 2012-2016}

\section{Computer Skills}
\label{sec:skills}
\cvlistitem{\textbf{Programming Languages}: C/C++, Python, Matlab, Java}
\cvlistitem{\textbf{Libraries:}:  OpenCV, Tensorflow, Pytorch}



% \section{Service}
% \label{sec:service}

% \subsection{Statistics outreach}
% \cvline{}{I've worked to advance statistical literacy among cognitive scientists, psychologists, and linguists, via presentations at workshops and tutorials on statistical methods.}
% \cvline{2014}{Tutorial on simulation-based methods for understanding mixed models. \emph{SST 2014 Tutorial: Regression and mixed effects models}}
% \cvline{2013}{Co-organized UR BCS Department statistics and data analysis working group with Steve Piantadosi.}
% \cvline{2011}{\emph{Preparing data for analysis using R}, presentation with Alex B. Fine at University of Rochester BCS workshop on mixed-effects regression analysis for cognitive scientists.}
% \cvline{2010}{\emph{ANOVA and Mixed-Effects Models in R for Language Scientists}, presentation at University of Maryland Linguistics Dept. Winter Storm language sciences workshop.}

% \subsection{Open source software}

% \cvline{}{I've contributed to a number of open source software projects.  I've made substantial contributions to:}
% \cvlistitem{\textbf{\href{https://github.com/JuliaStats/}{JuliaStats}}: Statistical computing in the \href{http://julialang.org/}{Julia language}.  Contributed model formula parsing and evalutaion domain-specific language code to regression modeling functionality in \href{https://github.com/JuliaStats/DataFrames.jl}{DataFrames.jl} and \href{https://github.com/JuliaStats/GLM.jl}{GLM.jl}.}
% \cvlistitem{\textbf{\href{https://github.com/dcjones/Gadfly.jl}{Gadfly.jl}}: Data visualization in Julia, based on the grammar of graphics.}

% \cvline{}{And contributed minor bug fixes and improvements to \textbf{\href{https://github.com/ftilmann/latexdiff/}{latexdiff}}, \textbf{\href{https://github.com/mroth/lolcommits}{lolcommits}}, a node.js \textbf{\href{https://github.com/jefftimesten/mturk}{mturk API}}, and \textbf{\href{https://github.com/JoFrhwld/FAAV}{FAAV/FAVE}}.}

% \subsection{Reviewing}

% \cvline{}{Ad hoc reivewer for
%   \emph{Acta Psychologica};
%   \emph{Annual Meeting of the Cognitive Science Society};
%   \emph{Attention, Perception, \& Psychophysics};
%   \emph{Cognition};
%   \emph{Journal of the Acoustical Society of America};
%   \emph{Journal of Experimental Psychology};
%   \emph{Journal of Memory \& Language};
%   \emph{Language};
%   \emph{Language and Cognitive Processes};
%   \emph{MIT Press};
%   \emph{NSF SBE};
%   \emph{PLoS One}; and
%   \emph{Topics in Cognitive Science}
% }

% \subsection{Mentoring}

% \cvline{2012}{Supervised three undergraduate independent study students (jointly with one other graduate student).}

\end{document}


